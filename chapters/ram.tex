\subsection{Complessità delle RAM}

\subsubsection{Criteri di Costo}
\begin{itemize}
    \item \textbf{Costo Costante:} Ogni istruzione ha costo 1. Ogni cella di memoria ha costo 1, indipendentemente dal valore contenuto.
    \item \textbf{Costo Logaritmico:} Il costo di un'operazione e dello spazio occupato dipende dalla dimensione (logaritmo) dei valori numerici coinvolti.
    $$
    l(x) := 
    \begin{cases}
        \lfloor \log_2 x \rfloor + 1 & \text{se } x \neq 0 \\
        1 & \text{se } x = 0
    \end{cases}
    \quad \text{N.B.: } l(x) = \Theta(\log x)
    $$
    \item \textbf{Quando sceglierli:} I due criteri sono equivalenti se la dimensione degli operandi è limitata da una costante. Se i numeri possono diventare arbitrariamente grandi, il criterio logaritmico è più realistico.
\end{itemize}

\subsubsection{Calcolo del Costo Logaritmico (caso semplificato)}
Sotto l'ipotesi di usare un numero costante di celle di memoria:
\begin{itemize}
    \item \textbf{Costo Spaziale:} Lo spazio totale è la somma della "lunghezza" (logaritmo) di tutti i numeri più grandi salvati in ogni cella di memoria utilizzata. E' sempre $\Theta(\log i)$, dove $i$ è il numero che viene calcolato in quell'istante.
    \item \textbf{Gestione di un intero i} (es. \texttt{LOAD, STORE, READ, WRITE, JZ})
    \begin{itemize}
        \item Costo Temporale: $\Theta(\log i)$
    \end{itemize}
    \item \textbf{Operazioni Aritmetiche} (su operandi $n_1, n_2$)
    \begin{itemize}
        \item Addizione (+), Sottrazione (-): $\Theta(\log n_1 + \log n_2)$
        \item Moltiplicazione (*), Divisione (/): $\Theta(\log n_1 \cdot \log n_2)$
    \end{itemize}
\end{itemize}

Come si calcola in generale il costo temporale in caso di costo algoritmico? In generale si prende l'operazione nell'utlimo istante, che può essere magari una somma o una moltiplicazione, e con una sommatoria si somma tutto. Usi l'approssimazione di Stirling per calcolare la complessità.
