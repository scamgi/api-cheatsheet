\subsection{Sintassi di base}

\begin{itemize}
    \item \textbf{Riga di commento}
    \begin{lstlisting}
//...
    \end{lstlisting}
    \item \textbf{Assegnamento}
    \begin{lstlisting}
i := j
    \end{lstlisting}
    \item \textbf{Operazioni}
    \begin{lstlisting}
+, -, *, /, %
    \end{lstlisting}
    \item \textbf{Confronto di interi}
    \begin{lstlisting}
>, <, >=, <=, =, !=
    \end{lstlisting}
    \item \textbf{Lettura dell'input}
    \begin{lstlisting}
x := read()
    \end{lstlisting}
    \item \textbf{Restituzione in output}
    \begin{lstlisting}
return x
    \end{lstlisting}
\end{itemize}

\subsection{Istruzioni comuni}

\begin{itemize}
    \item \textbf{If-else}
    \begin{lstlisting}
if condizione
    istruzioni
else
    istruzioni
    \end{lstlisting}
    \item \textbf{Cicli}
    \begin{lstlisting}
while condizione
    istruzioni

for i := n_1 to n_2
    istruzioni
    \end{lstlisting}
\end{itemize}

\subsection{Oggetti e variabili}

\begin{itemize}
    \item I dati composti sono organizzati come oggetti. Gli oggetti hanno attributi (campi):
    \begin{itemize}
        \item \texttt{x.attr} è il valore dell'attributo \texttt{attr} dell'oggetto \texttt{x}.
    \end{itemize}
    \item Gli array sono oggetti, dotati dell'attributo \texttt{length}.
    \begin{itemize}
        \item \(A[j]\) è l'elemento di indice \(j\) dell'array A.
        \item \(A[i..j]\) è il sottoarray di A dall'i-esimo al j-esimo elemento.
    \end{itemize}
    \item Una variabile che corrisponde ad un oggetto è un puntatore all'oggetto.
    \begin{itemize}
        \item Dopo le istruzioni \( y := x, x.attr := 3 \) si ha \( y.attr = x.attr = 3 \).
    \end{itemize}
    \item Un puntatore che non fa riferimento ad alcun oggetto ha valore NIL.
\end{itemize}