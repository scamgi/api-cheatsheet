\subsection{Vettori (Array)}
\begin{itemize}
    \item \texttt{A.length} = lunghezza dell'array.
    \item \texttt{A[i]} = accesso a elemento \texttt{i} dell'array.
    \item \texttt{A[i..j]} = sottoarray da \texttt{i} a \texttt{j}.
    \item \texttt{n} = dimensione dell'array che è uguale a \texttt{A.length}.
\end{itemize}

\subsection{Matrici}
\begin{itemize}
    \item \texttt{M.height} = numero di righe.
    \item \texttt{M.width} = numero di colonne.
    \item \texttt{M[i][j]} = accesso a riga i colonna j.
    \item \texttt{n} = dimensione dell'input che è uguale al numero di elementi, ovvero $M.\texttt{height} \times M.\texttt{width}$.
    \item $n := M.\texttt{size}$ per una matrice quadrata, dove \texttt{size} è il numero di righe (o colonne).
\end{itemize}

\subsection{Liste Concatenate}
\begin{itemize}
    \item \texttt{L.head} = puntatore alla testa della lista.
    \item \texttt{x\_f.next = NIL}, dove \texttt{x\_f} è l'ultimo elemento della lista.
\end{itemize}

\subsubsection{Liste Singolarmente Concatenate}
\begin{itemize}
    \item \texttt{x.key} = dato contenuto nell'elemento $x$.
    \item \texttt{x.next} = puntatore all'elemento successivo.
\end{itemize}

\subsubsection{Liste Doppiamente Concatenate}
\begin{itemize}
    \item \texttt{x.key} = dato contenuto nell'elemento $x$.
    \item \texttt{x.next} = puntatore all'elemento successivo.
    \item \texttt{x.prev} = puntatore all'elemento precedente.
    \item \texttt{L.head.prev = NIL}.
\end{itemize}

\subsubsection{Liste Doppiamente Concatenate Circolari}
\begin{itemize}
    \item Utilizzano un nodo speciale detto \textbf{sentinella} (\texttt{L.nil}) al posto di \texttt{L.head}.
    \item \texttt{L.nil.key = NIL}.
    \item \texttt{L.nil.next} punta alla testa della lista.
    \item \texttt{L.nil.prev} punta alla coda della lista.
    \item La lista è "circolare": la \texttt{prev} della testa e la \texttt{next} della coda puntano a \texttt{L.nil}.
    \item Se la lista è vuota, \texttt{L.nil} punta a se stesso.
\end{itemize}

\subsection{Tabelle Hash}
\begin{itemize}
    \item \texttt{T} = array di \texttt{m} celle (slot) che memorizza i dati.
    \item \texttt{h(k)} = funzione hash che mappa una chiave \texttt{k} a un indice della tabella.
    \item \texttt{$\alpha=n/m$} = fattore di carico, definito come rapporto tra elementi e slot.
\end{itemize}

\subsection{Alberi}

\subsubsection{Alberi Binari di Ricerca (BST)}
\begin{itemize}
    \item \texttt{T.root} = puntatore alla radice dell'albero.
    \item \texttt{x.key} = chiave del nodo \texttt{x}.
    \item \texttt{x.p} = puntatore al nodo padre.
    \item \texttt{x.left} = puntatore al figlio sinistro.
    \item \texttt{x.right} = puntatore al figlio destro.
    \item \texttt{x.leftsize} = (opzionale) dimensione del sottoalbero sinistro del nodo.
\end{itemize}

% TODO aggiungi gli Algoritmi di attraversamento
% TODO aggiungi i Dynamic set operations

\subsubsection{Alberi di Ricerca Generici (GST)}
\begin{itemize}
    \item \texttt{T.root} = puntatore alla radice dell'albero.
    \item \texttt{x.key} = chiave del nodo \texttt{x}.
    \item \texttt{x.p} = puntatore al nodo padre.
    \item \texttt{x.fs} = puntatore al figlio più a sinistra (first son).
    \item \texttt{x.lb} = puntatore al fratello a sinistra (left brother).
    \item \texttt{x.rb} = puntatore al fratello a destra (right brother).
\end{itemize}

\subsubsection{Alberi Rosso-Neri (RBT)}
\begin{itemize}
    \item Un RBT è un BST con attributi e proprietà aggiuntive.
    \item \texttt{T.nil} = nodo speciale \textbf{sentinella} che sostituisce i puntatori a \texttt{NIL}. Il suo colore è sempre \texttt{BLACK}.
    \item \texttt{x.color} = attributo di ogni nodo che può essere \texttt{RED} o \texttt{BLACK}.
    \item \texttt{bh(x)} = \textbf{altezza nera} del nodo, ovvero il numero di nodi neri in ogni cammino da \texttt{x} (escluso) a \texttt{T.nil} (incluso).
\end{itemize}

\paragraph{Proprietà RB}
\begin{itemize}
    \item Ogni nodo è \textbf{rosso o nero}.
    \item La \textbf{radice è nera} (\texttt{T.root.color = BLACK}).
    \item Ogni foglia (il nodo sentinella \texttt{T.nil}) è \textbf{nera}.
    \item Se un nodo è rosso, allora entrambi i suoi \textbf{figli sono neri}.
    \item Per ogni nodo, tutti i cammini semplici da quel nodo alle foglie discendenti contengono lo \textbf{stesso numero di nodi neri}.
\end{itemize}