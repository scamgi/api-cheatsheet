\subsection{Complessità degli Automi}
\begin{itemize}
    \item \textbf{DFSA (Automa a Stati Finiti Deterministico)}
    \begin{itemize}
        \item Complessità Temporale: $T_A(n) = \Theta(n)$
        \item Complessità Spaziale: $S_A(n) = \Theta(1)$
    \end{itemize}

    \item \textbf{DPDA (Automa a Pila Deterministico)}
    \begin{itemize}
        \item Complessità Temporale: $T_A(n) = \Theta(n)$
        \item Complessità Spaziale: $\Theta(0) \le \Theta(S_A(n)) \le \Theta(n)$
    \end{itemize}

    \item \textbf{k-DTM (Macchina di Turing Deterministica a k-nastri)}
    \begin{itemize}
        \item Complessità Temporale: Nessun limite generale. Per calcolarla devi immaginare il funzionamento della macchina.
        \item Complessità Spaziale: $\Theta(S_M(n)) \le \Theta(T_M(n))$
    \end{itemize}

    \item \textbf{SDTM (Macchina di Turing Deterministica a nastro singolo)}
    \begin{itemize}
        \item Complessità Temporale: Nessun limite generale. Per calcolarla devi immaginare il funzionamento della macchina.
        \item Complessità Spaziale: $S_M(n) = \Omega(n)$, ciò significa che la complessità spaziale dev'essere almeno lineare, questo perché il nastro di input coincide con il nastro di memoria.
    \end{itemize}
\end{itemize}

\textbf{TIP}: per il calcolo della complessità spaziale, ricordati di considerare il caso peggiore. Il caso peggiore può anche essere per una stringa che non viene accettata, ovvero $x \notin L$.

\subsection{Contatori (Implementati su DTM)}
\begin{itemize}
    \item \textbf{Complessità Spaziale}: per contare fino a $m$, sono necessari $\Theta(\log m)$ simboli. Se $m$ dipende dalla lunghezza dell'input $n$, la complessità spaziale diventa $S_M(n) = \Theta(\log n)$.
    \item \textbf{Complessità Temporale} (per eseguire $n$ incrementi/decrementi):
    \begin{itemize}
        \item $T(n) = \Theta(n)$: se ad ogni modifica vengono visitate solo le cifre necessarie (userai questo in sede d'esame).
        \item $T(n) = \Theta(n \log n)$: se ad ogni modifica si visitano tutte le cifre del contatore.
    \end{itemize}
\end{itemize}