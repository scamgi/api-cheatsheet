\subsection{Algoritmi comuni}

\begin{center}
    \begin{tabular}{|l|c|c|}
        \hline
        \textbf{Algoritmo} & \textbf{Complessità temporale} & \textbf{Complessità spaziale} \\
        \hline
        Insertion sort & $\Theta(n^2)$ & $\Theta(1)$ \\
        \hline
        Merge sort & $\Theta(n \log n)$ & $\Theta(n)$ \\
        \hline
        Heapsort & $\Theta(n \log n)$ & $\Theta(1)$ \\
        \hline
        Quicksort & $\Theta(n^2)$ & $\Theta(1)$ \\
        \hline
        Counting sort & $\Theta(n+k)$ & $\Theta(k)$ \\
        \hline
    \end{tabular}
\end{center}

Algoritmi di ricerca:

\begin{itemize}
    \item \texttt{LIN-SEARCH}: ha complessità $\Theta(n)$
    \item \texttt{BIN-SEARCH}: ha complessità $\Theta(n)$
\end{itemize}

\subsection{Come Riconoscere la Complessità Logaritmica}

Un algoritmo ha complessità logaritmica quando il problema si riduce in modo esponenziale a ogni passo. I principali indizi nel codice sono:

\begin{itemize}
    \item \textbf{La variabile del ciclo moltiplica o divide.}
        \begin{itemize}
            \item L'aggiornamento non è un'addizione/sottrazione (es. \texttt{i := i + 1}).
            \item L'aggiornamento è una moltiplicazione o divisione per una costante $c > 1$ (es. \texttt{i := i * 2} oppure \texttt{i := i / 2}).
            \item La variabile "salta" verso il valore finale, invece di "camminare". Questo richiede $\Theta(\log n)$ passi.
        \end{itemize}
    
    \item \textbf{Lo spazio del problema viene ridotto di una frazione costante.}
        \begin{itemize}
            \item L'algoritmo scarta una porzione significativa dei dati a ogni iterazione (es. metà, un terzo, etc.).
            \item L'esempio classico è la Ricerca Binaria, che dimezza lo spazio di ricerca a ogni passo.
            \item La ricorrenza associata è spesso nella forma $T(n) = T(n/b) + O(1)$, la cui soluzione è $\Theta(\log n)$.
        \end{itemize}

    \item \textbf{Caso speciale ($\log \log n$): la variabile esegue un "super-salto".}
        \begin{itemize}
            \item La variabile di controllo viene elevata a una potenza, tipicamente al quadrato (es. \texttt{i := i * i}).
            \item La crescita è doppiamente esponenziale, portando a una complessità ancora minore di $\Theta(\log \log n)$.
        \end{itemize}
\end{itemize}
