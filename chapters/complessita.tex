\subsection{Notazioni di Complessità Asintotica in Elenco}

\begin{itemize}
    \item $f(n) = O(g(n))$ - \textbf{O grande} - Limite asintotico superiore
    \item $f(n) = \Omega(g(n))$ - \textbf{Omega grande} - Limite asintotico inferiore
    \item $f(n) = \Theta(g(n))$ - \textbf{Theta grande} - Limite asintotico sia superiore che inferiore
\end{itemize}

\subsection{Confronto Tramite Limiti}
Dato il limite $L = \lim_{n\to\infty} \frac{f(n)}{g(n)}$:
\begin{itemize}
    \item Se $L = 0$, allora $\Theta(f(n)) < \Theta(g(n))$.
    \item Se $L = c$ (con $c \neq 0, \infty$), allora $\Theta(f(n)) = \Theta(g(n))$.
    \item Se $L = \infty$, allora $\Theta(f(n)) > \Theta(g(n))$.
\end{itemize}

\subsection{Gerarchia Fondamentale degli Ordini di Grandezza}
Per costanti $k,h \in \mathbb{R}^+$ e $a>1$:
$$ \Theta(1) < \Theta((\log n)^{k}) < \Theta(n^{h}) < \Theta(a^{n}) < \Theta(n!) < \Theta(n^{n}) $$

\subsection{Classi di Complessità Comuni}
\begin{itemize}
    \item $\mathcal{O}(1)$: Costante (es. accesso a un elemento di un array)
    \item $\mathcal{O}(\log n)$: Logaritmica (es. ricerca binaria)
    \item $\mathcal{O}(n)$: Lineare (es. scansione di una lista)
    \item $\mathcal{O}(n \log n)$: Lineare-logaritmica (es. merge sort, heapsort)
    \item $\mathcal{O}(n^2)$: Quadratica (es. bubble sort, selection sort)
    \item $\mathcal{O}(2^n)$: Esponenziale (es. problemi risolti con la forza bruta)
    \item $\mathcal{O}(n!)$: Fattoriale (es. problema del commesso viaggiatore con forza bruta)
\end{itemize}