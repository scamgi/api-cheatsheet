\subsection{Notazioni di Complessità Asintotica in Elenco}

\begin{itemize}
    \item $f(n) = O(g(n))$ - \textbf{O grande} - Limite asintotico superiore
    \item $f(n) = \Omega(g(n))$ - \textbf{Omega grande} - Limite asintotico inferiore
    \item $f(n) = \Theta(g(n))$ - \textbf{Theta grande} - Limite asintotico sia superiore che inferiore
\end{itemize}

\subsection{Confronto Tramite Limiti}
Dato il limite $L = \lim_{n\to\infty} \frac{f(n)}{g(n)}$:
\begin{itemize}
    \item Se $L = 0$, allora $\Theta(f(n)) < \Theta(g(n))$.
    \item Se $L = c$ (con $c \neq 0, \infty$), allora $\Theta(f(n)) = \Theta(g(n))$.
    \item Se $L = \infty$, allora $\Theta(f(n)) > \Theta(g(n))$.
\end{itemize}

\subsection{Gerarchia Fondamentale degli Ordini di Grandezza}
Per costanti $k,h \in \mathbb{R}^+$ e $a>1$:
$$ \Theta(1) < \Theta((\log n)^{k}) < \Theta(n^{h}) < \Theta(a^{n}) < \Theta(n!) < \Theta(n^{n}) $$

\subsection{Complessità degli Automi}
\begin{itemize}
    \item \textbf{DFSA (Automa a Stati Finiti Deterministico)}
    \begin{itemize}
        \item Complessità Temporale: $T_A(n) = \Theta(n)$
        \item Complessità Spaziale: $S_A(n) = \Theta(1)$
    \end{itemize}

    \item \textbf{DPDA (Automa a Pila Deterministico)}
    \begin{itemize}
        \item Complessità Temporale: $T_A(n) = \Theta(n)$
        \item Complessità Spaziale: $\Theta(0) \le \Theta(S_A(n)) \le \Theta(n)$
    \end{itemize}

    \item \textbf{k-DTM (Macchina di Turing Deterministica a k-nastri)}
    \begin{itemize}
        \item Complessità Temporale: Nessun limite generale.
        \item Complessità Spaziale: $\Theta(S_M(n)) \le \Theta(T_M(n))$
    \end{itemize}

    \item \textbf{SDTM (Macchina di Turing Deterministica a nastro singolo)}
    \begin{itemize}
        \item Complessità Temporale: Nessun limite generale.
        \item Complessità Spaziale: $S_M(n) = \Omega(n)$
    \end{itemize}
\end{itemize}

\subsection{Complessità delle RAM}

\subsubsection{Criteri di Costo}
\begin{itemize}
    \item \textbf{Costo Costante:} Ogni istruzione ha costo 1. Ogni cella di memoria ha costo 1, indipendentemente dal valore contenuto.
    \item \textbf{Costo Logaritmico:} Il costo di un'operazione e dello spazio occupato dipende dalla dimensione (logaritmo) dei valori numerici coinvolti.
    $$
    l(x) := 
    \begin{cases}
        \lfloor \log_2 x \rfloor + 1 & \text{se } x \neq 0 \\
        1 & \text{se } x = 0
    \end{cases}
    \quad \text{N.B.: } l(x) = \Theta(\log x)
    $$
    \item \textbf{Quando sceglierli:} I due criteri sono equivalenti se la dimensione degli operandi è limitata da una costante. Se i numeri possono diventare arbitrariamente grandi, il criterio logaritmico è più realistico.
\end{itemize}

\subsubsection{Calcolo del Costo Logaritmico (caso semplificato)}
Sotto l'ipotesi di usare un numero costante di celle di memoria:
\begin{itemize}
    \item \textbf{Costo Spaziale:} Lo spazio totale è la somma della "lunghezza" (logaritmo) di tutti i numeri più grandi salvati in ogni cella di memoria utilizzata.
    \item \textbf{Gestione di un intero i} (es. \texttt{LOAD, STORE, READ, WRITE, JZ})
    \begin{itemize}
        \item Costo Temporale: $\Theta(\log i)$
    \end{itemize}
    \item \textbf{Operazioni Aritmetiche} (su operandi $n_1, n_2$)
    \begin{itemize}
        \item Addizione (+), Sottrazione (-): $\Theta(\log n_1 + \log n_2)$
        \item Moltiplicazione (*), Divisione (/): $\Theta(\log n_1 \cdot \log n_2)$
    \end{itemize}
\end{itemize}

\subsection{Classi di Complessità Comuni}
\begin{itemize}
    \item $\mathcal{O}(1)$: Costante (es. accesso a un elemento di un array)
    \item $\mathcal{O}(\log n)$: Logaritmica (es. ricerca binaria)
    \item $\mathcal{O}(n)$: Lineare (es. scansione di una lista)
    \item $\mathcal{O}(n \log n)$: Lineare-logaritmica (es. merge sort, heapsort)
    \item $\mathcal{O}(n^2)$: Quadratica (es. bubble sort, selection sort)
    \item $\mathcal{O}(2^n)$: Esponenziale (es. problemi risolti con la forza bruta)
    \item $\mathcal{O}(n!)$: Fattoriale (es. problema del commesso viaggiatore con forza bruta)
\end{itemize}