% --- Pacchetti e configurazioni di base ---
\documentclass[10pt, a4paper]{article}

% --- Lingua e codifica ---
\usepackage[utf8]{inputenc}
\usepackage[T1]{fontenc}
\usepackage[italian]{babel}

% --- Matematica e simboli ---
\usepackage{amsmath}         % Per ambienti matematici avanzati
\usepackage{amssymb}         % Per simboli matematici
\usepackage{amsfonts}        % Per i font matematici
\usepackage{xfrac}           % Per frazioni "slanted" (es. 1/2)

% --- Layout e stile ---
\usepackage[a4paper, margin=1cm]{geometry} % Margini ridotti per un cheatsheet
\usepackage{multicol}      % Per disporre il testo in più colonne
\usepackage{graphicx}      % Per includere immagini
\usepackage{parskip}       % Per avere spazio tra i paragrafi invece dell'indentazione
\usepackage{xcolor}        % Per usare i colori

% --- Link e riferimenti ---
\usepackage{hyperref}
\hypersetup{
    colorlinks=true,
    linkcolor=blue,
    filecolor=magenta,
    urlcolor=cyan,
}

% --- Comandi personalizzati (utili per la matematica) ---
\newcommand{\bigO}[1]{\mathcal{O}(#1)}
\newcommand{\bigOmega}[1]{\Omega(#1)}
\newcommand{\bigTheta}[1]{\Theta(#1)}

% --- Informazioni per il titolo ---
\title{Cheatsheet di Algoritmi e Strutture Dati}
\author{Giacomo Scampini} % Puoi inserire il tuo nome qui
\date{}  % Rimuove la data 